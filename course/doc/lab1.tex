\documentclass[a4paper,12pt]{article}

\usepackage[hidelinks]{hyperref}
\usepackage{amsmath}
\usepackage{mathtools}
\usepackage{shorttoc}
\usepackage{cmap}
\usepackage[T2A]{fontenc}
\usepackage[utf8]{inputenc}
\usepackage[english, russian]{babel}
\usepackage{xcolor}
\usepackage{graphicx}
\usepackage{float}
\graphicspath{{./img/}}

\definecolor{linkcolor}{HTML}{000000}
\definecolor{urlcolor}{HTML}{0085FF}
\hypersetup{pdfstartview=FitH,  linkcolor=linkcolor,urlcolor=urlcolor, colorlinks=true}

\DeclarePairedDelimiter{\floor}{\lfloor}{\rfloor}

\renewcommand*\contentsname{Содержание}

\newcommand{\plot}[3]{
    \begin{figure}[H]
        \begin{center}
            \includegraphics[scale=0.6]{#1}
            \caption{#2}
            \label{#3}
        \end{center}
    \end{figure}
}

\begin{document}


    \begin{titlepage}
	\begin{center}
		{\large Санкт-Петербургский политехнический университет\\Петра Великого\\}
	\end{center}
	
	\begin{center}
		{\large Физико-механический иститут}
	\end{center}
	
	
	\begin{center}
		{\large Кафедра «Прикладная математика»}
	\end{center}
	
	\vspace{8em}
	
	\begin{center}
		{\bfseries Отчёт по лабораторной работе №2 \\по дисциплине «Анализ данных с интервальной неопределённостью»}
	\end{center}
	
	\vspace{5em}
	
	\begin{flushleft}
		\hspace{16em}Выполнил студент:\\\hspace{16em}Габдрахманов Булат Маратович\\\hspace{16em}группа: 5040102/20201
		
		\vspace{2em}
		
		\hspace{16em}Проверил:\\\hspace{16em}к.ф.-м.н., доцент\\\hspace{16em}Баженов Александр Николаевич
		
	\end{flushleft}
	
	
	\vspace{6em}
	
	
	\begin{center}
		Санкт-Петербург\\2023 г.
	\end{center}	
	
\end{titlepage}
    \newpage

    \tableofcontents
    \listoffigures
    \newpage

    \section{Постановка задачи}
    \quad Необходимо провести анализ, используя несимметричные меры совместности интервалов.

    \section{Теория}
    \subsection{Бинарная мера совместности (покрытия) интервалов.}

    По мере развития интервального анализа, были введены различные
    определения и конструкции оценки меры совместности интервальных
    объектов. Всем известно, что вопрос сравнения интервалов рассмотрен
    в публикации В.М. Левина. Однако в этой работе рассмотрены
    только абсолютные меры сравнения.

    Вместе с тем, в практике обработки данных часто необходимо
    оперировать с относительными величинами.
    В частности, это нужно из-за необходимости соизмерения допусков и размеров деталей, потребности размеретилей и значений измеряемых величин, и т.п.

    В публикации S.Kabir et all, вводится различные меры сходства
    или объектов нечётких множеств и интервалов, используются обозначения $S(A, B)$.

    Кроме того, авторы вводят несимметричную меру \textit{overlapping ratio}
    для интервала $I_i$ относительно пары интервалов $\{I_i, I_j\}$

    \begin{equation}
    OR(I_i, I_j) = \frac{|I_i \cap I_j|}{|I_i|}\label{eq:equation1}
    \end{equation}

    \begin{equation}
    s = \min(OR(I_i, I_j), OR(I_j, I_i)).
    \end{equation}

    Мера \textit{несходства}, \textit{dissimilarity} соответственно имеет вид $d = 1 - s$.

    Все меры, представленные в перечисленных работах, развивают
    идею Жаккара, и оперируют только с пересекающимися интервалами.
    В публикации обсуждаются также и непересекающиеся интервалы,
    но вычисления меры совместности не производится.

    \subsection{Обобщение бинарной меры совместности (покрытия) интервалов.}
    Приступим к обобщению меры совместности интервалов, имея
    целью описывать единообразно как накрывающие, так и не накрывающие
    выборки.
    Для анализа данных необходимо проводить сравнение интервальных объектов универсальным образом, независимо от свойств накрытия.
    В первую очередь, введём базовую конструкцию совместности для двух интервалов. Для иллюстрации идеи, рассмотрим следующую числовую характеристику степени совместности двух интервалов \(x, y\):

    \begin{equation}
    \text{wid} \left( \frac{x \land y}{x \lor y} \right)
    \end{equation}

    В выражении в качестве абсолютной меры величины сходства используется ширина интервала, а вместо операции пересечения и объединения множеств — операции взятия минимума (\(\land\)) и максимума (\(\lor\)) по включению двух величин в полной интервальной арифметике (Каухера).

    В общем случае, минимум по включению в выражении может быть неправильным интервалом.
    При этом его ширина не определена и нужно использовать либо его правильную проекцию, либо задать нужную конструкцию в явном виде. В связи с этим, введём относительную меру покрытия как

    \begin{equation}
    JK(x, y) = \frac{\min\{\overline{x}, \overline{y}\} - \max\{\underline{x}, \underline{y}\}}{\max\{\overline{x}, \overline{y}\} - \min\{\underline{x}, \underline{y}\}}
    \end{equation}

    В записи формулы, вместо ширины интервалов используются выражения взятия минимума и максимума по включению, обеспечивающие универсальный характер конструкции, независимо от того, является ли результат операции взятия минимума по включению (\(\land\)) правильным или неправильным интервалом.

    Альтернативным способом записи может быть использование в числителе ширины \(\text{wid} (x \land y)\), с учётом правильности или неправильности полученного интервала знаком.

    \begin{equation}
    JK(x, y) = \frac{\text{wid} (x \land y)}{\text{wid} (x \lor y)}
    \end{equation}

    Автору представляется использование вышеуказанных выражений более предпочтительным виду дальнейшего обобщения на выборки интервальных величин и общего минимаксного духа арифметики Каухера.
    В именовании \(JK(x, y)\) буква J отвечает фамилии Jaccard, а K указывает на арифметику Каухера.
    \textbf{Несимметричные варианты меры JK.} Мера симметрична относительно своих аргументов.
    Это может оказаться неудобным в случае большой разницы ширин аргументов. В таком случае могут оказаться полезными несимметричные варианты.

    \[
    s_x(x, y) = \frac{\min\{\overline{x}, \overline{y}\} - \max\{\underline{x}, \underline{y}\}}{\text{wid } x}, \quad s_y(x, y) = \frac{\min\{\overline{x}, \overline{y}\} - \max\{\underline{x}, \underline{y}\}}{\text{wid } y}
    \]

    Подобные конструкции именуют мера схождства (<<similarity>>), <<overlapping ratio>>.
    Выражение отличается от них тем, что допускает неравильность интервала при взятии минимума по включению.

    \textbf{Пример 1 (Вычисление несимметричных мер сходства)} Пусть \( x = [1, 2] \), \( y = [3, 7] \).

    Вычислим меры

    \[
    s_x(x, y) = \frac{2 - 3}{1} = -1, \quad s_y(x, y) = \frac{2 - 3}{4} = -\frac{1}{4}
    \]

    Абсолютная величина отражает равенство <<несходства>> операндов относительно ширины первого из них. При этом величина существенно меньше ширины 2-го операнда.
    Этот факт соответствует разнице ширин операндов и позволяет строить на этой основе содержательные конструкции, например, при процедурах регуляризации.

    \section{Изотопная подпись для некоторых типов органических соединений.}\label{sec:------.}

    \begin{table}[h]
    \centering
    \caption{Вариации изотопного состава сульфидов в единицах \(10^3 \cdot \delta^{34}S_{VCDT}\)}
    \label{tab:sulfide}
    \begin{tabular}{|l|l|l|}
    \hline
    Подкатегория & Нижняя граница & Верхняя граница \\
    \hline
    Atmospheric H2S & -32 & 20 \\
    \hline
    Thermogenic H2S & 0 & 30 \\
    \hline
    Surface water/ground water & -55 & 30 \\
    \hline
    Minerals & -54 & 70 \\
    \hline
    Reagent H2S & 1 & 23 \\
    \hline
    Other reagents & -33 & 23 \\
    \hline
    \end{tabular}
    \end{table}
    \plot{sulfide_intervals}{Вариации изотопного состава сульфидов в единицах \(10^3 \cdot \delta^{34}S_{VCDT}\)}{p:sulfide_interv}

    \[
    X = 10^3 \cdot \delta^{34}S_{VCDT} = \left\{
    \begin{aligned}
    &[-32, 20], &[0, 30], &[-55, 30], &[-54, 70], &[1, 23], &[-33, 23]
    \end{aligned}
    \right\}
    \]

    Интервальная мода для выборки равна
    \[
    \text{mode}(10^3 \cdot \delta^{13}C_{VCDT}) = [1, 20].
    \]
    Интервальная мода — это такой подинтервал \(z_i\), который является подмножеством всех интервалов \(x_j\), входящих в выборку: \(z_i \subseteq x_j\).
    \plot{sulfide_mode}{Интервальная мода вариаций \(10^3 \cdot \delta^{34}S_{VCDT}\)}{p:sulfide_mod}

    Как видно из численных данных и Рис. 2, ширина интервальной моды является небольшой по отношению к ширине всей выборки.

    Для характеризации выборки применим индекс Жаккара:
    \[
    J_i(X) = \frac{\text{wid}(\Lambda_i; x_i)}{\text{wid}(V; x_i)} = -0.036.
    \]
    Величина отрицательна. Причина состоит в том, что величина \(x_2 = [-16, -9]\) имеет пустое пересечение с модой (5.3) \([1, 20]\). Это указание на то, что характер изотопной метаболического процесса подписи сульфида иной, чем в выбранной категории веществ.

    \section{Пример}
    Интервальная мода для выборки сульфидов равна
    \[
    \text{mode}(10^3 \cdot \delta^{34}S_{VCDT}) = [1, 20].
    \]

    Рассмотрим меры сходства для всех подкатегорий сульфидов:

    \begin{itemize}
    \item Для атмосферного H2S с интервалами значений \( x = [-32, 20] \):
    \[
    J_{ix}(x, y) = \frac{\min\{x,y\} - \max\{x,y\}}{\text{wid } x} = \frac{\min\{20, 20\} - \max\{-32, 1\}}{52} = \frac{19}{52} \approx 0.3654.
    \]

    \item Для термогенного H2S с интервалами значений \( x = [0, 30] \):
    \[
    J_{ix}(x, y) = \frac{\min\{x,y\} - \max\{x,y\}}{\text{wid } x} = \frac{\min\{20, 30\} - \max\{0, 1\}}{30} = \frac{19}{30} \approx 0.6333.
    \]

    \item Для поверхностной воды/грунтовых вод с интервалами значений \( x = [-55, 30] \):
    \[
    J_{ix}(x, y) = \frac{\min\{x,y\} - \max\{x,y\}}{\text{wid } x} = \frac{\min\{20, 30\} - \max\{-55, 1\}}{85} = \frac{19}{85} \approx 0.2235.
    \]

    \item Для минералов с интервалами значений \( x = [-54, 70] \):
    \[
    J_{ix}(x, y) = \frac{\min\{x,y\} - \max\{x,y\}}{\text{wid } x} = \frac{\min\{20, 70\} - \max\{-54, 1\}}{124} = \frac{19}{124} \approx 0.1532.
    \]

    \item Для реагентного H2S с интервалами значений \( x = [1, 23] \):
    \[
    J_{ix}(x, y) = \frac{\min\{x,y\} - \max\{x,y\}}{\text{wid } x} = \frac{\min\{20, 23\} - \max\{1, 1\}}{22} = \frac{19}{22} \approx 0.8636.
    \]

    \item Для других реагентов с интервалами значений \( x = [-33, 23] \):
    \[
    J_{ix}(x, y) = \frac{\min\{x,y\} - \max\{x,y\}}{\text{wid } x} = \frac{\min\{20, 23\} - \max\{-33, 1\}}{56} = \frac{19}{56} \approx 0.3393.
    \]
    \end{itemize}

    Как видно из расчетов, различные подкатегории сульфидов имеют разную степень перекрытия с интервальной модой.

    \begin{table}[h]
    \centering
    \caption{Меры сходства с интервальной модой для выборки сульфидов}
    \label{tab:sulfide_similarity}
    \begin{tabular}{|l|c|c|}
    \hline
    Категория & Интервал & \( J_{ix} \) \\
    \hline
    Атмосферный H2S & \([-32, 20]\) & 0.365 \\
    \hline
    Термогенный H2S & \([0, 30]\) & 0.633 \\
    \hline
    Поверхностная вода/грунтовые воды & \([-55, 30]\) & 0.224 \\
    \hline
    Минералы & \([-54, 70]\) & 0.153 \\
    \hline
    Реагентный H2S & \([1, 23]\) & 0.864 \\
    \hline
    Другие реагенты & \([-33, 23]\) & 0.339 \\
    \hline
    \end{tabular}
    \end{table}
    \newpage

    \section{Заключение}\label{sec:}

    В данной работе были рассмотрены различные подкатегории сульфидов и их сходство с заданной интервальной модой. Исследование показало, что меры сходства могут значительно варьироваться в зависимости от характеристик конкретной подкатегории. Атмосферный H2S и реагентный H2S показали наибольшее сходство с интервальной модой, в то время как термогенный H2S и минералы демонстрировали отрицательное значение, что указывает на отсутствие пересечения с модой.

    Результаты исследования могут быть использованы для глубокого понимания изотопного состава сульфидов в различных средах и условиях их формирования. Это, в свою очередь, может способствовать разработке новых методов анализа и применения сульфидов в различных областях, от геологических исследований до промышленного применения.

    \end{document}
